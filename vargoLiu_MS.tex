Turning up the Heat: improving the Specificity of Heat Warning Systems

Motivation
The frequency, duration, areal coverage, and intensity of heat waves are expected to increase for most populated places because of global climate change. Such events have caused major episodic mortality – including an estimated 70,000 premature(?) deaths in Europe 2003 and 20,000(?) in Russia 2010. In many places, early warning systems are a key, if not the primary, component of measures for avoiding heat related deaths and illness. In the United States the National Weather Service (NWS) issues such warnings. In addition to shortcomings of the warning system mentioned previously, we are concerned with the spatial resolution and specificity of the issued warnings. The predictions for urban areas, in particular could be improved given that cities are typically hotter than surrounding areas, concentrate, people and demonstrate great variability in temperatures and adaptive capacities. 

In this work we consider temperatures from distributed weather stations, NWS heat advisories, and land surface temperature imagery throughout two large metropolitan areas, Atlanta and Chicago, during the summers from 2006-2012. We first investigate the spatial variability in hazardous temperatures and agreement with advisories issued by the NWS and second examine the potential for a widely available thermal imagery product to replicate National Weather Service heat warnings.

Background
The relationship between heat and health is well-understood, and known to change over time and by location and population.  The expectations for future changes to climate are directly related to expected health impacts {Kinney2008}. The relationship between heat and health has generally been investigated in two distinct ways: first, by investigating statistical extremes and, second, by examining specific events. Several studies using large population data sets and several summers of data demonstrate a positive relationship between temperature and excess all-cause mortality {Medina, Basu, Currerio} with varying thresholds based on region. 

Heat response planning was only formalized in relatively recent efforts, primarily in cities like Chicago and Philadelphia following deadly heat waves, in 1995 and _____ respectively {CITE}. The events led to greater attention on individual and population level factors affecting heat vulnerability. These include age, social isolation, housing type, income, and ethnicity. Resulting efforts including vulnerable population censuses and door-to-door visits during heat events have been shown to be successful in reducing the related excess mortality {Li, et al.}. The increased prevalence of energy-intensive adaptation strategies, most notably, air conditioning is another reason for declining heat-related mortality in the US in spite of increased frequency of such events {DAVIS2003}. Finally, predictive weather models with increased accuracy and timeliness have improved the forecasting of hot air masses moving and settling over regions. 

The potential for future extreme heat events that pose danger to human health is increasing with global climate change. As James Hansen and colleagues at NASA’s Goddard Institute for Space Studies put it, we have been loading the climate dice for the last 30 years, increasing the likelihood of meteorological events that used to occur once ever hundred years. Models and recent historical records echo this sentiment with data and the recent news that atmospheric CO2 concentrations passed 400ppm for the first time in 3 million years imply that these trends are likely to continue.  Early and accurate warnings are a crucial component for adapting to more frequent hazardous heat events.

Currently the NWS uses four products based on the heat index (HI), a value combining temperature and relative humidity to describe heat stress and discomfort, to issue hazard notifications. Estimates of temperatures and humidity are predicted days in advance by National weather models which are now available at a spatial resolution of 2.5 km. 

Excessive Heat Outlook: may be issued 3 to 7 days prior to a heat episode requiring issuance of a Heat Warning, provided forecaster confidence is relatively high
Excessive Heat Watch: may be issued 12 to 48 hours prior to heat episode with a 50 percent chance or greater of daytime heat indices equal to or greater than 110 degrees for at least two consecutive days.
Heat Advisory: is issued in the first and/or second period when there is an 80 percent chance or greater of daytime heat indices equal to or greater than 105 degrees for at least two consecutive days.
Excessive Heat Warning: is issued in the first and/or second period when there is an 80 percent or greater of daytime heat indices equal to or greater than 110 degrees for at least two consecutive days.
Some of the main criticisms of the NWS’s advisory guidelines relate to their uniformity for places across the country. In this respect they have not been representative of regional behavioral adaptation practices. This leads to warnings and advisories in some areas that may be issued too frequently or not frequently enough and do not represent actual weather-related risks for people. In both cases the issuances may lead to skepticism and neglect for the NWS notifications and decrease their effectiveness. In recent years some regions, like Chicago, have been more proactive about defining their own criteria not only for daytime high HI as well as nighttime low HI. 


Other concerns with the NWS advisories and warnings relate to urban areas where anthropogenic land cover modifications may elevate temperatures dangerously high even outside periods of regional extreme weather. Hazardous thermal conditions may also exist in small pockets of the urban landscape, which displays great variability in temperatures over space.  The current NWS notification system is based on large grid climate model and releases warnings and advisories at the county level. Because counties vary in size from place to place and may or may not wholly contain major population centers, the NWS products may fail to provide specific warnings and meaningful guidance for large numbers of people. Even if the products are accurately predicting regional hot weather, detail to the spatial distribution of temperatures within the region could improve warnings, individual adaptation measures, and larger response efforts. 

Previous work has examined whether the HI values used are indicative of health threats by comparing HI values and mortality for various locations. Like the NWS products the HI values are often collected at a single site used as representative for the entire region. The NWS products themselves have also been compared with mortality and morbidity rates in places. The outcome of such studies has shown that HI can be a useful metric for informing decisions to protect public health in some locations. Alternatives based on weather classifications, like the Synoptic Spatial Classification (SSC), have also been examined for agreement with variability in mortality. The NWS system has also been compared against public perceptions of heat risk, a determining factor in whether precautionary actions are taken, and found to vary between regions and across age, income, gender, and ethnicities within regions. 

NEW DATA SOURCES
In situ and remote sensors of temperature at or near the Earth’s surface offer the potential to improve the current warning system by increasing its specificity, particularly with regard to location. These data sources are new and numerous. They provide data with greater spatial coverage, frequent revisit times, and low cost. Distributed weather stations may have a shorter history than reliable long term stations used in climatological studies, but can effectively describe spatial patterns in temperatures across the urban landscape.  Imagery products like MODIS’s land surface temperature (LST) offer 1km spatial resolution data for areas around the globe twice daily. Such products are systematically processed to remove inconsistencies and distortions, as well as to calculate usable temperature outputs. While these data can not currently offer predictive weather information, they are nonetheless useful for analyzing and describing spatial variation and patterns which, when combined with predictive model currently in use, can improve information to the public and the adaptive response. 

To our knowledge, we are the first to consider spatial variability in the NWS heat warning system inside of the regional or county boundary, and the first to compare the National advisory system to land surface temperatures 

